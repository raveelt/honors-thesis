\documentclass[twocolumn,showpacs,preprintnumbers,amsmath,amssymb]{revtex4}
%\documentclass[preprint,showpacs,preprintnumbers,amsmath,amssymb]{revtex4}
% Additional packages needed for graphics, alignment and math fonts.

\usepackage{graphicx}% Enhanced capability for dealing with figures
\usepackage{dcolumn}% Align table columns on decimal point.
\usepackage{bm}% bold math.
% The \begin{document} command sets the start of the RevTeX commands.
	
	\begin{document}
		
		\title{A quantum algorithm for the bottleneck travelling salesman problem}
		
		\author{Raveel Tejani}
		\email{raveel@student.ubc.ca}
		\affiliation{Department of Physics and Astronomy, University of British
			Columbia \\
			6224 Agricultural Road, Vancouver, British Columbia, V6T 1Z1, Canada}
		
		\date{\today}
		
		\begin{abstract}
			%statement of what will be accomplished, the methods that will be used and
			%the potential significance.  Many of you have already sent me a description
			%of your project that can be used as the basis for this.  I will only accept
			%\LaTeX\ complied files in pdf format (and we're going to specifically use
			%revtex, the style for Physical Review). They must be uploaded by midnight
			%before the oral presentation is scheduled.  Make sure your supervisor has
			%a chance to read it over before you submit it.
			
			the Bottleneck Travelling Salesman Problem is an NP-Hard optimization problem. The implication being
			there exists no solution in polynomial time.  We propose an algorithm to provide a 
			quadratic speedup using the grover's quantum search algorthim as the basis. We will test this algorithm on either
			real quantum hardware or simulators, implement improvements and discuss our findings.
			
		\end{abstract}
		
		\maketitle
		
		% The first section of the paper.
		
		\section{Motivation}
		
		
		Continuous advancements in quantum computing have opened up new ways to solve complex computational problems that were previously considered intractable using classical methods. The Bottleneck Traveling Salesman Problem (BTSP), is a good example of an optimization problem that stands as a challenge in the field of logistics and operations research (need ref). Its practical applications range from optimizing delivery routes to circuit design (need ref). As a result, achieving an efficient solution is highly sought after.
		
		We embark on a journey at the intersection of quantum computing and optimization by introducing a quantum algorithm tailored to address the BTSP. It requires finding the shortest closed tour through a set of cities while minimizing the largest cost (the "bottleneck") along any route. Its computational complexity grows exponentially with the number of cities (need ref), rendering classical solutions impractical for large-scale instances.
		
		Our approach is to capitalize on the inherent quantum parallelism, as well as the unique feature of phase encoding in quantum computing. By exploiting these phenomena, we aim to encode and manipulate the costs associated with various routes efficiently. Additionally, the utilization of Grover's search algorithm as a central component of our quantum approach significantly expedites the search for the optimal solution.
		
		In this proposal, our primary objective is to shed light on the theoretical foundation of the quantum algorithm, its potential computational advantages over classical methods. Through this research, we aim to work towards a clear understanding of these aspects. The following sections will detail the BTSP, its expected computational complexity, Grover's algorithm and the significance of its implementation.
		
		\textbf{This section in the examples was typically a page, needs more substance. Maybe elaborating on the practical examples a little more. Dont love last paragraph} 
		
		
		\section{Theory}
		
		\subsection{Bottleneck Travelling Salesman Problem}
		
		\begin{figure}[!h]
			\centering
			\includegraphics[trim={0 0 21.9cm 0},clip, width=0.7 \linewidth]{"../4-city graphic"}
			\caption{An undirected weighted graph represention of a  symmetric 4-city system.  The vertices represent cities and the edge weights represent the cost of travel. }
			\label{fig:4-city-graphic}
		\end{figure}		
		
		
		
		
		The BTSP can be represented as a graph problem.  We start with a graph, whose vertices are labelled A through D, representing a 4-city system. We define movement from one vertex to another as a walk, done through the edges connecting our vertices. We are interested in a particular walk known as a hamiltonian cycle that contains every vertex extactly once before returning to the start. Our graph also includes edge weights, we define as $ \gamma_i $.
		The BTSP is to find the hamiltonian cycle in a graph, where the largest edge weight ("bottleneck") is minimized. This is distinct from the Travelling Salesman Problem (TSP) where the combined edge weights in a given cycle is minimized. The total possible hamiltonian cycles is given by $(N-1)!$, where N is the number of nodes. We present a symmetric case in the figure,  $N_k \rightarrow N_{k+1} = N_{k+1} \rightarrow N_{k} = \gamma_i$. Thus the total possible cycles is  $(N-1)!/2$.  It is important to note that a solution to either BTSP or TSP is not unique. BTSP solutions also do not nessesarily equate to the TSP solutions. We can illustrate an example with the help of our figure. Consider all the hamiltonian cycles for a symmetric 4-city system:
		
		\textbf{the discussion of hamiltonian cycles within the paragraph could maybe be a little more formal? seperating the math from the paragraph?} 
		
		\begin{center}
		$ A \rightarrow B \rightarrow C \rightarrow D \rightarrow A $
		
		$ A \rightarrow B \rightarrow D \rightarrow C \rightarrow A $ 
		
		$ A \rightarrow C \rightarrow B \rightarrow D \rightarrow A $
	    \end{center}
		
		Assigning some arbritrary weights, we can see the total costs of the cycles below. The first cycle is the solution to BTSP as its largest edge weight at $5$ is the smallest among all three. The last cycle is a solution to the TSP as its combined edge weight is the smallest.
		\begin{center}
		$\gamma_1 + \gamma_4 + \gamma_5 + \gamma_3 = 4 + 4 + 5 + 4 = 17$
		
		$ \gamma_1 + \gamma_6 + \gamma_5 + \gamma_2 = 4 + 6 + 5 + 2 = 17$
		
		$  \gamma_2 + \gamma_4 + \gamma_6 + \gamma_3 = 2 + 4 + 6 + 4 = 16$
		\end{center}
		

		By simply changing the weight of $\gamma_6$ to $5$, we can illustrate all cycles are solutions to the BTSP. 
		
		\begin{center}

			$\gamma_1 + \gamma_4 + \gamma_5 + \gamma_3 = 4 + 4 + 5 + 4 = 17$
			
			$ \gamma_1 + \gamma_6 + \gamma_5 + \gamma_2 = 4 + 5 + 5 + 2 = 16$
			
			$  \gamma_2 + \gamma_4 + \gamma_6 + \gamma_3 = 2 + 4 + 5 + 4 = 15$
		\end{center}
		
		The computational complexity of the BTSP is shown to be NP-hard (reference.). Implying there is no algorithm for a solution in polynomial time. A brute-force approach would imply that we can run an algorithm in $O((N-1)!)$ time
		
		\subsection{Grover's Search Algorithm}
		
		\begin{figure}[!h]
			\centering
			\includegraphics[trim={5cm 26cm 15cm 0},clip, width=0.99 \linewidth]{../grover_circuit}
			\caption{A concise quantum circuit representation of Grover's search algorithm applied to an n-qubit system. The Hadamard $H^{\otimes n }$ gate performs a uniform superposition over all possible states $N$ ($2^n$). The $U_w$ gate marks our goal state. The $U_s$ gate amplifies the amplitude of our goal state. the $r$ exponent, refers to the number of iterations needed to maximize the amplitude. The approximate number is $\sqrt{N}$ times.}
			\label{fig:grovercircuit}
		\end{figure}
		
		In 1996, Lov Grover proposed a quantum algorithm for unstructured database search, that would provide at most a quadractic speedup (need ref). This speed up can be realized for suffiently large databases, but does not provide the exponential speedup promised by other quantum algorithms. We can refer to figure 2, to see the quantum circuit representation.
		A geometric representation of the algorithm as well as the two-qubit example will help us understand. 
		
		
		\begin{figure}[!h]
			\centering
			\includegraphics[trim={0 7.5cm 0 0},clip, width=0.99\linewidth]{../grover_geometric}
			\caption{geometric reprentation of Grover's algorithm.  a) An iteration performed on an n-qubit system.  Starting from the uniform superposition $|s\rangle$, we perform a reflection over the orthogonal states of $|w\rangle$ with $U_w$. Then we perform a reflection over the superposition state $|s\rangle$ with $U_s$. we can see our new state $|x\rangle = U_sU_w|s\rangle$ is closer to the goal state $|w\rangle$. A new iteration, would imply reflecting again over $|w_\perp\rangle$ and then reflecting back over $|x\rangle$. b) An iteration performed on a 2-qubit system. Only one iteration is required to hit any goal state $|w\rangle$ }
			\label{fig:grovergeometric}. 
		\end{figure}
		
		
		\textbf{do i really need a two - qubit example in the figure? I dont refer to it in the text below at all. The kets are also not generating properly in fig 2, 3. Need to fix}
		
		Lets walk through one iteration of Grover's algorithm. Most quantum circuits are initialized at $|0\rangle$, then we apply the $H$ gate, this results in a uniform superposition:
		
		$$ |s\rangle = H^{\otimes n} |0 \rangle = \frac{1}{\sqrt{2^n}} \sum_{x\in\{0,1\}^n} |x\rangle$$
		
	%	$$ |s\rangle_2 = H^{\otimes 2} |0 \rangle = \frac{1}{2} (|00\rangle + |01\rangle + |10\rangle + |11\rangle)$$
		
		The orthogonal states  to our goal state can be represented as the sum of all states not containing $|w\rangle$
		
		$$ |w_\perp\rangle=  \frac{1}{\sqrt{2^n -1}} \sum_{x\in\{0,1\}^n}^{x \neq w} |x\rangle$$
		
		From here we apply the $U_w$ gate, for our reflection over the orthogonal states. This can be thought of flipping the sign onstate $|w\rangle$ contained within $|s\rangle$. We can see the visual representation in figure 3. We then perform a similiar reflection over the original state $|s\rangle$ :
		
		$$ U_w|s \rangle	=  (\mathbb {I} - 2|w \rangle \langle w|) |s \rangle = |s'\rangle$$
		$$ U_s|s' \rangle	=  (2|s \rangle \langle s| - \mathbb {I}) |s' \rangle = |s''\rangle$$
		
		Our $|s''\rangle$ is now closer to our goal state $|w\rangle$.  If we want to maximize our probability of measuring  $|w\rangle$, we will need to perform a number of iterations ($r$). For sufficiently large number of states, $N = 2^n$, we can make a few approximations to calculate $r$
		\begin{equation}
			\langle w | s \rangle = \frac{1}{\sqrt{N}} = sin \frac{\theta}{2} \approx \frac{\theta}{2}
		\end{equation}
		
		
		$$P(w) = |\langle w | s \rangle|^2 = \frac{1}{N} = sin^2 \frac{\theta}{2} $$
		
		$$P(w) = |\langle w |(U_sU_w)^r |s \rangle|^2 = \frac{1}{N} = sin^2 (\theta(\frac{1}{2} + r)) $$
		
		to maximize the probability we need P(w) = 1, so we can set the inside of the sine function to $\pi/2$:
		$$ \theta \left(\frac{1}{2} + r \right) = \frac{\pi}{2}$$
		
		\begin{equation}
			r = \frac{\pi}{2\theta} + \frac{1}{2} \approx \frac{\pi}{2\theta}
		\end{equation}
		
		Using eqn. 1 to substitute $\theta$ in eqn 3, we can get an approximate result for r
		
		\begin{equation}
			r \approx \frac{\pi}{4}\sqrt{N}
		\end{equation}
		
		
		
		 %A simple example of minimizing the bottleneck travelling salesman problem (maybe with images):
		%an example of storing phases and using phase estimation
		%grover's search explanation
		
		\section{details of proposed experiment/calculation}
		
		The goal is to implement a quantum algorithm that will calculate a solution for the BTSP in $\mathcal{O}(\sqrt{(N-1)!})$
		
		The challenge of this project is constructing $U_w$, this is sometimes referred to as the oracle. We gave a geometric explanation in the theory section, but lets discuss the structure of an oracle in depth. Given a function $f: \{0,1,...,N-1\} \rightarrow \{0,1\}$:
		
		\begin{equation}
			U_{w}|x\rangle = (-1)^{f(x)}|x\rangle = 
			\begin{cases}
				-|x\rangle & \text{for  $x=w$ }\\
				|x\rangle  & \text{for $x\neq w$}\\
			\end{cases}       
		\end{equation}
		
		\textbf{how would this oracle change, since there are no unique solutions as illustrated in the theory section?}
		
		\textbf{this section needs a lot more work, hoping it is more in depth once i understand QFT / phase estimation. This would imply the theory section would grow as well.}
	
		\textbf{the discussion of $U_w$ might be better suited in the theory section?}
		
		%We need to encode our max edge costs per cycle into the domain of our function f(x). We then need to construct a matrix, $U_w$, that it will result in a $f(x) = 1$
		
		\section{recources list}
		\begin{enumerate}
			\item IBM Quantum Experience (10 min/month free access)
			\item Quantum Simulators
		\end{enumerate}
		
		\textbf{might not need this section, dont need recource request to conduct my thesis}
		\section{planned schedule}
		
		
		\begin{enumerate}
			\item October to December:  Oracle construction
			\item January to February:  Running it on a quantum system or simulator with analysis of results and implementing any improvements.
			\item  March: Thesis writing
		\end{enumerate}
		
		
		
		\section{Acknowledgements}
		
		I want to express my gratitude to Dr. Matthew Choptuik for not only supervising me but also 
		for his encouragement and support to pursue a topic that truly captivates my interest.
		
		\begin{thebibliography}{99}
			
			\textbf{No references listed yet}
			
		\end{thebibliography}
		
		
	\end{document}